\documentclass[12pt]{article}
\usepackage{graphicx,times,verbatim,amsmath,amssymb,amsthm}
\usepackage{multirow}
\usepackage{natbib}
\usepackage{subcaption}
\usepackage{color}

\newtheorem{theorem}{Theorem}[section]
\newtheorem{lemma}[theorem]{Lemma}
\newtheorem{corollary}[theorem]{Corollary}
\newtheorem{proposition}[theorem]{Proposition}
\newtheorem{definition}[theorem]{Definition}
\newtheorem{remark}[theorem]{Remark}
\newtheorem{condition}[theorem]{Condition}
\newtheorem{assumption}[theorem]{Assumption}
\newtheorem{property}[theorem]{Property}
\newtheorem{result}[theorem]{Result}
\newtheorem{example}[theorem]{Example}

\newtheorem*{remark*}{Remark}

\newcommand{\shrink}{\def\baselinestretch{0.90}\large\normalsize}
\newcommand{\thirdlevelsection}[1]{\vskip 4pt\noindent{\bf #1.}}
%\renewcommand{\includegraphics}{includegraphics}

\setlength{\oddsidemargin}{0in}
\setlength{\textwidth}{6.5 in}
\setlength{\textheight}{9. in}
\setlength{\topmargin}{-0.35in}
\setlength{\headsep}{0.0in}
%\setlength{\itemindent}{-2.5in}


\pagestyle{empty}

\newenvironment{packed_item}{
\begin{itemize}
  \setlength{\itemindent}{-0em}
  \setlength{\itemsep}{-0em}
  \setlength{\parskip}{-0em}
  \setlength{\parsep}{-0em}
}{\end{itemize}}
\newenvironment{packed_enum}{
\begin{enumerate}
  \setlength{\itemsep}{0pt}
  \setlength{\parskip}{0pt}
  \setlength{\parsep}{0pt}
}{\end{enumerate}}

\DeclareMathOperator*{\argmin}{arg\,min}
\DeclareMathOperator*{\argmax}{arg\,max}

\newcommand{\lmin}{\ell_{\min}}
\newcommand{\cfour}{\| \sum_{i=1}^n S_i \|_\infty}
\newcommand{\vertiii}[1]{{\left\vert\kern-0.25ex\left\vert\kern-0.25ex\left\vert #1
    \right\vert\kern-0.25ex\right\vert\kern-0.25ex\right\vert}}
\newcommand{\mean}[1]{\mkern 1.5mu \overline{\mkern-2mu #1 \mkern-2mu}\mkern 1.5mu}
\newcommand{\thickhline}{%
    \noalign {\ifnum 0=`}\fi \hrule height 1pt
    \futurelet \reserved@a \@xhline
}

\begin{document}

%\renewcommand{\baselinestretch}{1.1}


%\shrink
\pagestyle{plain}
\pagenumbering{arabic}
\normalbaselineskip=10.0pt



\begin{center}
Broadening Participation Plan
\end{center}

{\bf [This paragraph is the NSF instruction.]}
Describe the proposed institute's plan to increasing diversity, broadening participation, and encouraging involvement of underrepresented groups; howthis plan will be implemented, including identification of resources in the budget to support it; and how its outcomes will be measured. BroadeningParticipation plans should describe context, prior history of training activities, and concrete plans for action and evaluation. The BroadeningParticipation Plan may not exceed 5 pages total.


\subsection{Diversity and inclusion}

The leadership of TRIAD is committed to diversity and inclusion.
Co-PI Randall is the ADVANCE Professor of Computing at Georgia Tech, running workshops, interacting with campus leadership, faculty and students to ensure equitable policies on campus, and fostering mentoring and transparency.
Co-PI Tetali has developed partnership with Minority Serving Institutions, including Morehouse and Spelman; they have been sending undergraduate students to the school of mathematics (SOM) at GT each summer.
Kuske, the new chair of SOM, has served as the senior advisor to the provost (of the university of British Columbia) on women faculty, prior to arriving at GT. The committee for program diversity described in Collaboration and Evaluation Plan will further ensure TRIAD's commitment to diversity and inclusion.

\subsection{International collaboration supporting student and visitor exchange}

Prof. Koltchinskii, a senior person on this project, will spearhead efforts to create a joint center (as a part of TRIAD) with groups in Europe located in Paris, in Cambridge, at ETH (Switzerland) and at the Weierstrass Institute (Germany). Another senior person,
Prof. Popescu, will contribute by connecting TRIAD to researchers from Toulouse (France), Bonn (Germany), Oxford (U.K.) and Tokyo (Japan) along with the Tokyo Institute of Technology
in Chicago. One of our goals here is to support a PhD student exchange program with the universities in Europe, besides supporting  visitor programs. TRIAD  will seek joint ventures to realize the aforementioned.

\subsection{Marketing  capabilities and outputs to stakeholders and the public}
\label{sec:marketing}

The major vehicle for communicating the activities and successes of TRIAD to the public and stakeholders is the TRIAD web portal.
It will be built and maintained by experienced web designers and developers at Georgia Tech.
The aforementioned Communications Director will write news articles for the web portal, to be used in bimonthly newsletters.
Thus, in addition to the usual components of mission, people, activities, etc. of an organization website, TRIAD will have cutting edge articles on the newest research and education tools and technologies and their uses in homeland security.
Material for articles will come from the researchers themselves, directors of education projects, and semiannual reports.
The web portal
, overseen by the Communications Director,
will feature a team-fed content calendar and pipeline to ensure timely and comprehensive coverage of project activities.
It will include features such as a news section with press releases and news items, FAQs, downloadable resources, a form to sign up for TRIAD community newsletter, a calendar of events, a blog providing notable project updates and expert guest posts, and links to videos.
There will also be publications of institute members, technical reports, and other items downloadable by researchers outside of TRIAD.
Lists will contain useful data to allow segmentation by interest or audience to minimize off-target messaging and unsubscribes.
New members will be able to sign up using an online form easily located on the website.
Writing support will be drawn from Georgia Tech's associated colleges, Institute Communication, and occasionally from freelance writers.
As with GT's other large projects, TRIAD will maintain a list of relevant trade journals and respected outlets, and work to place content visible to fellow researchers, stakeholders, and the public.
We will leverage the network of Georgia Tech Institute Communications and other media-related connections to pitch and place major stories, develop targeted marketing, and generate awareness of important events, and will take advantage of Georgia Tech's Offices of Federal Affairs.
The institute will develop fact sheets, brochures, a social media presence, aimed at different audiences to be defined through the Communications Director's efforts.
Press releases and media events will also be organized and promoted by the Communications Director.

Annual reports will be provided and used as a source of news and publicity.
In addition, the Communications Director will oversee the construction of quarterly reports, which serve to inform the team, stakeholders, and community, and feed into the annual report.
Metrics concerning all communication endeavors will be routinely collected and assessed.




\end{document}


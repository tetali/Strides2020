\documentclass[12pt]{article}
\usepackage{graphicx,times,verbatim,amsmath,amssymb,amsthm}
\usepackage{multirow}
\usepackage{natbib}
\usepackage{subcaption}
\usepackage{color}

\newtheorem{theorem}{Theorem}[section]
\newtheorem{lemma}[theorem]{Lemma}
\newtheorem{corollary}[theorem]{Corollary}
\newtheorem{proposition}[theorem]{Proposition}
\newtheorem{definition}[theorem]{Definition}
\newtheorem{remark}[theorem]{Remark}
\newtheorem{condition}[theorem]{Condition}
\newtheorem{assumption}[theorem]{Assumption}
\newtheorem{property}[theorem]{Property}
\newtheorem{result}[theorem]{Result}
\newtheorem{example}[theorem]{Example}

\newtheorem*{remark*}{Remark}

\newcommand{\shrink}{\def\baselinestretch{0.90}\large\normalsize}
\newcommand{\thirdlevelsection}[1]{\vskip 4pt\noindent{\bf #1.}}
%\renewcommand{\includegraphics}{includegraphics}

\setlength{\oddsidemargin}{0in}
\setlength{\textwidth}{6.5 in}
\setlength{\textheight}{9. in}
\setlength{\topmargin}{-0.35in}
\setlength{\headsep}{0.0in}
%\setlength{\itemindent}{-2.5in}


\pagestyle{empty}

\newenvironment{packed_item}{
\begin{itemize}
  \setlength{\itemindent}{-0em}
  \setlength{\itemsep}{-0em}
  \setlength{\parskip}{-0em}
  \setlength{\parsep}{-0em}
}{\end{itemize}}
\newenvironment{packed_enum}{
\begin{enumerate}
  \setlength{\itemsep}{0pt}
  \setlength{\parskip}{0pt}
  \setlength{\parsep}{0pt}
}{\end{enumerate}}

\DeclareMathOperator*{\argmin}{arg\,min}
\DeclareMathOperator*{\argmax}{arg\,max}

\newcommand{\lmin}{\ell_{\min}}
\newcommand{\cfour}{\| \sum_{i=1}^n S_i \|_\infty}
\newcommand{\vertiii}[1]{{\left\vert\kern-0.25ex\left\vert\kern-0.25ex\left\vert #1
    \right\vert\kern-0.25ex\right\vert\kern-0.25ex\right\vert}}
\newcommand{\mean}[1]{\mkern 1.5mu \overline{\mkern-2mu #1 \mkern-2mu}\mkern 1.5mu}
\newcommand{\thickhline}{%
    \noalign {\ifnum 0=`}\fi \hrule height 1pt
    \futurelet \reserved@a \@xhline
}

\begin{document}

%\renewcommand{\baselinestretch}{1.1}


\clearpage
%\shrink
\pagestyle{plain}
\pagenumbering{arabic}
\normalbaselineskip=10.0pt

\begin{center}
Governance Plan
\end{center}

{\bf [This paragraph is the NSF instruction.]}
Describe the governance structure of the proposed institute, including a list of individuals who have agreed to serve as members of a governing boardor advisory council; mechanisms for fiscal and management oversight by a governing board or other group; plans for governing/advisory boardmembership terms and succession; mechanisms for focusing the proposed institute's activities; mechanisms for choosing programs, selectingparticipants, and allocating funds; mechanisms for recruitment, selection, and appointment involved in institute leadership succession and otherleadership changes; and rationales for the proposed management practices. The Governance Plan may not exceed 5 pages total.


\section{Operational Plan - [from TRIAD, need reworking]}
\label{sec:leadership}

TRIAD aims to have an agile and flexible agenda to be able to respond quickly to changing needs of the frontiers of the foundation of data science.
Achieving this requires regular evaluation.
Our leadership will have the responsibility of evaluating all institute programs as well as its own performance.
In addition, we will solicit assistance and guidance from national experts by constituting
an External Advisory Board (see Collaboration and Evaluation Plan for details).
Evaluation will focus on the timeliness and appropriateness of the research topics,
institute's activities to reach external audience, ways programs can improve, and on the institute's success in meeting its mission.

\vspace*{-1em}

\subsection{Leadership Structure for Phase I}
With full operation in Phase II in mind, we propose to ramp up operations with a more modest leadership structure for Phase I.
A detailed TRIAD leadership and management plan can be found in Sections 2.1 and 2.2 in the Collaboration and Evaluation plan.
A summary is presented below.

1) Lead PI Huo will serve as the Executive Director of the TRIAD institute.

2) Executive Committee: The TRIAD institute will be managed by its Executive Committee consisting of the five PIs, who exemplify the three foundational areas and community efforts.

3) Education and Training Committee: Responsible for curriculum development and educational outreach.

4) Committee for Program Diversity: Responsible for ensuring scientific as well as human diversity in all of the institute's operations.

5) Advisory Committees, including
\noindent \vspace*{-0.5em}
\begin{packed_item}
\item Member Council: The Council will initially consist of all senior personnel associated with the project, and is subject to future adjustment;

\item External Advisory Board: We will recruit top experts nationally for providing guidance, strategic advice, and independent evaluation.
\end{packed_item}

The Institute will be based at Georgia Tech, a world-class research university with extensive expertise in mathematics, theoretical computer science, and statistics.
Research and education at Georgia Tech is known for its real-world focus, and strong ties to government and industry.
It is located in Atlanta, the eighth largest economy in the nation.
Georgia Tech has a strong engineering presence: all of its computing and engineering departments are ranked in the top $10$ by the U.S. News \& World Report, with over half in the top five.
It is the largest producer of engineering degrees awarded to women and under-represented minorities.

Georgia Tech has shown very strong commitments to data science through several major investments in recent years.
In $2015$, Georgia Tech became one of the co-leads for the NSF South Big Data Hub, one of four in the country.
In $2016$, Georgia Tech launched the interdisciplinary Institute for Data Engineering and Science (IDEaS).
Georgia Tech is also the major partner behind the construction of Coda, a $750,000$ mixed-use facility in Technology Square next to the Georgia Tech campus.
Coda will be occupied equally by industry and academia when completed in $2018$, and will serve as the ideal location for TRIAD, as IDEaS, the NSF South Big Data Hub, and the School of Computational Science and Engineering will be co-located there.




\subsection{Building on the host institution's resources}
\label{sec:leverage}
Georgia Tech offers world class facilities to support national scope institutes and hosting
external visitors. Tech's Global Learning Center is a state-of-the art conference facility that can support meetings from small workshops to major conventions. The Center is adjacent
to the Georgia Tech Hotel and Convention Center, both on campus, providing easy access. Georgia Tech Strategic Consulting offers a range of services such as organizational review; strategy development and implementation; portfolio management; project management; change management; organizational design and development; and process optimization, that we can
leverage in establishing TRIAD. The Office of Industry Engagement assists with industry
partnerships, and with commercialization. Georgia Tech Institute Communications provides communications, marketing, and special events support. The communications personnel working
for other organizations (e.g., IDEaS) are hooked into this network, allowing TRIAD to benefit
from their professional expertise. Synergistic institutes on campus such as the IDEaS, South
Big Data Hub, Machine Learning center, etc. provide a holistic atmosphere to promote foundational data science research. TRIAD institute will be ultimately housed along with all of these in the new 21-story Coda building devoted to data science and high performance computing, in early 2019.
It evaluates promising technologies, markets Georgia Tech innovations to industry, negotiates licensing agreements, and maintains relationships with industry partners.


The Office of Sponsored Programs provides support and administration for all research contracts/grants at Georgia Tech.
, including review of contract/grant requirements, determination of appropriate terms and conditions, establishment of pre-contract agreements, and monitoring all active contracts and grants.
The Office of Sponsored Program provides assistance in identifying, reviewing, processing, and submitting formal proposals, and other services for business matters and contract terms.





\end{document}


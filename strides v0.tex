\documentclass[12pt]{article}
\usepackage{graphicx,times,verbatim,amsmath,amssymb,amsthm}
\usepackage{multirow}
\usepackage{natbib}
\usepackage{subcaption}
\usepackage{color}

\newtheorem{theorem}{Theorem}[section]
\newtheorem{lemma}[theorem]{Lemma}
\newtheorem{corollary}[theorem]{Corollary}
\newtheorem{proposition}[theorem]{Proposition}
\newtheorem{definition}[theorem]{Definition}
\newtheorem{remark}[theorem]{Remark}
\newtheorem{condition}[theorem]{Condition}
\newtheorem{assumption}[theorem]{Assumption}
\newtheorem{property}[theorem]{Property}
\newtheorem{result}[theorem]{Result}
\newtheorem{example}[theorem]{Example}

\newtheorem*{remark*}{Remark}

\newcommand{\shrink}{\def\baselinestretch{0.90}\large\normalsize}
\newcommand{\thirdlevelsection}[1]{\vskip 4pt\noindent{\bf #1.}}
%\renewcommand{\includegraphics}{includegraphics}

\setlength{\oddsidemargin}{0in}
\setlength{\textwidth}{6.5 in}
\setlength{\textheight}{9. in}
\setlength{\topmargin}{-0.35in}
\setlength{\headsep}{0.0in}
%\setlength{\itemindent}{-2.5in}

\pagestyle{plain} \pagenumbering{arabic}

\title{\bf NSF TRIPODS Phase II NSF 19-604  \\
\vspace*{0.5in}
Collaborative Research: STRIDES: Southeastern Transdisciplinary Research Institute for Data Engineering and Science
}

\author{The GT\&Duke STRIDES Team}

\pagestyle{empty}

\newenvironment{packed_item}{
\begin{itemize}
  \setlength{\itemindent}{-0em}
  \setlength{\itemsep}{-0em}
  \setlength{\parskip}{-0em}
  \setlength{\parsep}{-0em}
}{\end{itemize}}
\newenvironment{packed_enum}{
\begin{enumerate}
  \setlength{\itemsep}{0pt}
  \setlength{\parskip}{0pt}
  \setlength{\parsep}{0pt}
}{\end{enumerate}}

\DeclareMathOperator*{\argmin}{arg\,min}
\DeclareMathOperator*{\argmax}{arg\,max}

\newcommand{\lmin}{\ell_{\min}}
\newcommand{\cfour}{\| \sum_{i=1}^n S_i \|_\infty}
\newcommand{\vertiii}[1]{{\left\vert\kern-0.25ex\left\vert\kern-0.25ex\left\vert #1
    \right\vert\kern-0.25ex\right\vert\kern-0.25ex\right\vert}}
\newcommand{\mean}[1]{\mkern 1.5mu \overline{\mkern-2mu #1 \mkern-2mu}\mkern 1.5mu}
\newcommand{\thickhline}{%
    \noalign {\ifnum 0=`}\fi \hrule height 1pt
    \futurelet \reserved@a \@xhline
}

\begin{document}

%\renewcommand{\baselinestretch}{1.1}

\maketitle

\clearpage
%\shrink
\pagestyle{plain}
\pagenumbering{arabic}
\normalbaselineskip=10.0pt

\newcommand{\myspace}{1.3}
\baselineskip=\myspace\normalbaselineskip

\begin{center}
Summary
\end{center}

{\bf [this is from the LOI]}
We propose creating a new Institute aiming to explore and advance the synergies between mathematics, theoretical computer science (TCS), statistics, and electrical engineering in laying the theoretical foundations for data science. Proposed activities include national interdisciplinary workshops, collaborative cross-disciplinary projects, research labs with participants from all communities, joint seminars, and co-supervision of Ph.D. students and postdocs by multiple mentors and institutes, in order to catalyze and promote a true synergy among all related disciplines. Georgia Tech and Duke University will build on the existing strengths and prominence in all the research areas (mathematics, TCS, statistics, and electrical engineering) and build on synergistic avenues of research. Georgia Tech has renowned faculty in the College of Computing, the College of Engineering, the College of Sciences, and the Scheller College of Business and is considered the preeminent technology university in the Southeast. Duke is known as an institution for its interdisciplinarity, and Duke faculty and Duke faculty have world class expertise and recognition in data science, in addition Duke also has access to rich data repositories, including those of a top medical school and regional hospital as well as a 21st century social science data infrastructure built by the Social Science Research Institute (SSRI). The coupling of the strengths in the technological aspects of data science at Georgia Tech with the liberal arts, social science, and health aspects of data science at Duke offer a rich educational and research foundation in the Southeast. An important component of the proposal is to foster professional development both for junior data scientists as well as domain experts that want to become more involved in data science. Georgia Tech and Duke will develop innovative programs that will reach out to technology
centers, medical centers, as well as social science centers with the goal of increasing the interactions between junior data scientists and domain experts. The two institutions will also promote diversity and representation that reflects the community via collaborations with Under-represented Minorities institutions.

Georgia Tech and Duke have already made significant investments. Georgia Tech has a new Interdisciplinary Research
Institute on Data Engineering and Science (IDEaS), a new 21-story building (Coda) co-locating data science industry and academia in Midtown Atlanta, and serving as collaborative lead institution to the NSF South Big Data Hub, which spans 16 Southern states and over a hundred partner organizations. Duke University has made considerable investments in data science research and education through the Rhodes Information Initiative at Duke (iiD), initiated in 2013.
The iiD has features including labs, team rooms, classrooms, offices, and the 2-story Ahmadieh Family Atrium, all of which are designed to incubate cross-disciplinary interactions among faculty, postdocs, and students. In addition, Duke has a new data science initiative focusing on information and analysis-driven health innovation entitled AI.Health which has a goal of engaging the biotechnology and biomedical industries in the NC Triangle region
including the Duke Hospital system with data science expertise and developing the educational framework to train data scientists in this application space. Duke University and Georgia Tech are prominent research institutes in the southeast. Georgia Tech is conveniently located at the midtown of Atlanta, with easy access to public transportation (e.g., MARTA), which routes directly to Atlanta's international airport. The Hartsfield-Jackson Atlanta International
Airport serves the largest number of passengers in the world and offers direct flights to most US cities. Many Fortune 500 companies have headquarters in Atlanta. There are many local companies that are data oriented. Duke University is in the Research Triangle in North Carolina with strong local industrial research presence. Both Duke University and Georgia Tech have strengths in all related research communities. The joint institute (by Duke and Georgia Tech) will directly benefit the entire southeast, while also having a positive national impact.

\clearpage

{\bf NSF requirement.}
Project Description, limited to 30 pages total, consists of each of the following topics:

The {\bf intellectual focus of the proposed institute}; the rationale for the proposed institute, its mission and goals, and its expected impact; plans for future growth and resource development; proposed steps toward developing its role as a national resource; and results of prior NSF support of the institute if applicable. This section is not to exceed 20 pages total including results of prior NSF support, which may take up to 5 pages.

A tentative {\bf schedule} of scientific activities, with plans for Year 1 and a provisional schedule for Years 2 and 3.

{\bf Plans for human resource development}, including the selection and mentoring of a diverse cohort of students and postdoctoral participants, as appropriate, and the selection and involvement of researchers at all career levels.

{\bf Plans for outreach and for dissemination of outcomes}.

\clearpage
\setcounter{page}{1}

\begin{center}
{\bf\large I. Intellectual Focus of the Proposed Institute}
\end{center}
 (not to exceed 20 pages)

\section{Introduction}

%\vspace*{-1em}
The Georgia Institute of Technology and the Duke University propose to create {\it \bf STRIDES:} {\it the {\bf S}outheastern {\bf T}ransdisciplinary {\bf R}esearch {\bf I}nstitute for {\bf D}ata {\bf E}ngineering and {\bf S}cience}.
STRIDES will integrate research and education in mathematical, statistical, and algorithmic foundations for data science.
The initial research topics of focus include {\bf xxx the following will be updated later.}
(T1) transcribing data with new models and mathematics,
(T2) creating new paradigms for decentralized and scalable inference,
and (T3) designing efficient strategies with theoretical guarantees, harnessing the combined
perspectives of statistics, optimization, and numerical methods.
The research will be carried out in the context of big datasets from multiple domains
including biology, design, manufacturing, logistics, and sustainability.

During its lifetime, the institute will develop methods and tools to address
high-impact multidisciplinary foundational challenges in data science.  The institute
will bring together mathematicians, computer scientists, and statisticians to jointly
work with domain specialists with data challenges across engineering, science and computing.
Data arising from experimental, observational, and/or simulated processes in the natural and
social sciences and other areas have created enormous opportunities for understanding the
world in which we live. Data science is already a reality in industrial and scientific
enterprises and there is ever-increasing demand for both research and training in this field.
Virtually every scientific discipline is expected to benefit from advances in theoretical
foundations of data science, which we seek to strengthen through the STRIDES institute.

%{\it Fundamental problems in data science.}
The spectrum of fundamental problems in data science is vast. We broadly categorize them as
addressing 1) how the data is collected and interpreted, and 2) how the data is analyzed.
To address the former, we propose to study modeling and analysis techniques for data with new features and in nontraditional formats (T1), as well as develop decentralized modeling and
processing techniques which would not require the data to be transferred to a data center (T2).
To address the latter aspect, we propose to study modeling approaches with optimal statistical
and computational trade-offs, and develop algorithms that can be accelerated, distributed/parallelized, are asynchronous and/or stochastic (T3).
Key to fundamental advances in these research topics is to derive theoretical guarantees in both the asymptotic and finite-sample cases.

{\bf Intellectual merits.}
The emergence of massive computational power via cloud computing and supercomputing infrastructure
has given theorists an unprecedented opportunity to join the fray of empirical science and make
significant impact on applications. STRIDES will be particularly well placed to address the growing challenges in the foundations of data science. STRIDES's intellectual focus is to design and build
transdisciplinary research programs that provide an enabling and cross-fertilizing platform of ideas and stakeholders (including theoreticians/scientists from domain sciences and users of technology).


{\bf Broader impacts.}
STRIDES will enrich careers of participants ranging from undergraduate students to senior researchers from around the nation in due time. We will make prudent efforts to reach out to diverse
communities including participants from smaller colleges and institutions serving under-represented minorities. STRIDES team will actively engage in outreach through public lectures, curricular
materials, press releases and dissemination via social channels.

\vspace*{-1em}
\subsection{Key Functions of the Institute}
%{\it Research topic identification.}
STRIDES will lead and administer a fully integrated program of Research, Education and Outreach in foundations of data science.  The institute's research agenda will inform educational programs and
will be driven by transdisciplinary solutions to pressing data science challenges.  The institute will facilitate interactive interactions between theorists and practitioners with the aim of bringing fundamentally rigorous and broadly applicable solutions and tools to support data-driven discovery in sciences, engineering, and beyond.

%{\it Overview of proposed activities}.
Proposed activities include national interdisciplinary workshops, collaborative
cross-disciplinary projects, research labs with participants from all communities,
joint seminars, and co-supervision of Ph.D. students and postdocs by multiple mentors,
in order to catalyze and promote a true synergy among mathematicians, statisticians,
and theoretical and algorithmic computer scientists.

STRIDES is particularly well placed to address the growing challenges in data science.
Nowadays, analysis of massive, dynamic, and complex data is an area of great importance
in many domains. To support true understanding of what is feasible through data-driven
approaches and develop broadly applicable and insightful solutions, it is imperative to
establish theoretical foundations of data science. Much of the underlying intellectual
foundations underpinning data science lies at the intersection between computer science, statistics, and mathematics.
%STRIDES's intellectual focus is to design and build transdisciplinary research programs that provide an enabling and cross-fertilizing platform of ideas and stakeholders (including theoreticians/scientists from domain sciences and users of technology).
In Phase I, STRIDES's main mode of operation will be focused on creating and operating
working groups, organizing national and international workshops, and innovation labs.
Participating individuals will include senior, mid-career, and junior faculty members,
early career researchers including research scientists and postdoctoral fellows,
undergraduate and graduate students, and data science practitioners.
All activities will be planned to include interdisciplinary researchers rooted in the
three foundational disciplines: mathematics, theoretical computer science, and statistics.
STRIDES will rapidly deploy information technology and communication infrastructure so that
research findings can be quickly and effectively disseminated, while the research community
at large can easily access and comment/critique STRIDES's choice of research programs and topics.
The utilization of contemporary cyber-infrastructure is likely to lead to international impact
and collaboration, which enhances the institute's ability to establish a solid foundation for
data science research. The institute will create an intellectual atmosphere that connects
theoreticians, practitioners, and scientists from across the nation and the world on a
regular basis. Findings in STRIDES's activities will lead to presentations at major conferences
and publications in refereed journals.

%{\bf Broader impacts.}
STRIDES's programs will enrich careers of participants ranging from undergraduate students to senior researchers from around the nation.
Postdoctoral fellows and graduate students are introduced to collaborative research in the proposed activities and through workshops.
STRIDES personnel will make prudent efforts to reach out to diverse communities including
participants from smaller colleges and institutions serving under-represented minorities.
STRIDES will conduct public outreach through public lectures, press releases, and dissemination via internet and social media.
%STRIDES will work with associated professional societies, including Association for Computing Machinery,  American Statistical Association, American Mathematics Society, Society of Industrial and Applied Mathematics, IEEE, and potentially others to provide stimulus to data-science-related initiatives.
STRIDES will draw upon the nationally acclaimed Georgia Tech online degree programs (e.g., online
masters programs in computer science and in analytics) and curriculum development efforts in
machine learning and data science, so that students across the nation can learn about state-of-the-art and interdisciplinary research topics that are not typically covered within traditional campus courses.
For Phase II, we propose additional activities (such as customized workshops) that will combine
interactive projects and field trips to acquaint undergraduate and/or high-school students
nationwide with data-science related techniques and the themes of the STRIDES's year-long programs.


%{\it Our strengths.}
\vspace*{-1em}
\subsection{University Support and Infrastructure}
Georgia Institute of Technology (abbreviated Georgia Tech hereafter) is a world-class research university with extensive expertise in data analytics, statistics, theoretical
computer science, operations research and simulation, and mathematics.
All of its computing and engineering departments are ranked in the top 10 by the
U.S. News \& World Report, with over half in the top five. Most of its statistics, optimization,
and operations research faculty are housed in the School of Industrial Systems and Engineering,
ranked as the top such department in the nation. Georgia Tech is the largest producer of engineering degrees awarded to women and underrepresented minorities. Research and education
at Georgia Tech is known for its real-world focus, and strong ties to government and industry.


Georgia Tech has shown very strong commitment to Data Science through several major investments in recent years. In 2016, Georgia Tech launched the interdisciplinary Research Institute (IRI) for Data Engineering and Science (IDEaS), charged with facilitating, nurturing, and promoting data science research and data-driven discovery across campus. Georgia Tech has both on-campus and
on-line M.S. program in Analytics, and is launching a Ph.D. program in Machine Learning with
a sizable data science component. It is investing in a \$375 million, 21-story, 750,000+ sq. ft.
building (termed Coda) devoted to data science and high performance computing, along
with a 80,000 sq. ft. data center to host large-scale computing and data repositories.
Planned for early 2019 occupancy, the building will be equally shared by Georgia Tech
and relevant data science industry, promoting academia-industry interaction. The STRIDES
institute will be administratively structured within IDEaS (led by co-PIs Aluru and Randall),
enabling it to benefit from staff, infrastructure, space, and other resources provided
through IDEaS, amplifying the impact of TRIPODS Phase I funding.

Georgia Tech has established itself as a national leader in the data sciences. Since 2015, it is serving as the collaborative-lead institution for the NSF South Big Data Regional Innovation Hub (led by co-PI Aluru),  one of four such Hubs established to serve the nation.
In this role, it supports regional and
national-scale efforts in research, industry adoption, and training activities across 16 Southern
states from Delaware to Texas, and Washington D.C. The Hub has over 150 partner organizations drawn from academia, industry, government labs, and non-profits. STRIDES team members have been at the
forefront of the data science revolution, involved in the
White House, NITRID, NSF, NIH, DOE, and DARPA big data initiatives, as well as funding from key programs such as NSF Bigdata, NIH BD2K, and DARPA XDATA.


%{\it ** We refer to Section~?? for more information on the (future) interaction between IDEaS and STRIDES.}
%{\it Either here, or later in Section 4, we mention that there exist multiple programs on campus that facilitate interdisciplinary education and research; this is  a strong signal that several members of SOM, CS/COC,  ISYE (and the Business School) already interact closely with each other: Examples include (besides IDEaS) research centers such as ARC, ML@GT, GT-MAP, and educational programs such as a PhD in ACO, CSE, Bioinformatics, QBIoS and ML, and Masters in QCF and Statistics. **}

%{\it Our location.}
Georgia Tech is located in Atlanta, the eighth largest economy in the nation and third among
cities with the most Fortune 500 companies~\cite{geolounge2016}. With the world's busiest airport
and single-hop reachability to many destinations that Atlanta provides, Georgia Tech is
uniquely positioned to serve meeting needs of the research community such as workshops and short courses. The proposed STRIDES institute aligns with the university's strategic plan, is synergistic
with many new initiatives, and will be housed in the new Coda building alongside IDEaS and the South Hub.
\vspace*{-1em}

\section{Research Program}
\label{sec:proposed}

\vspace*{-1em}
\subsection{Optimal transport}
\label{sec:optimal-transport}



\vspace*{-1em}
\subsection{Optimization related}
\label{sec:optimization-related}



\vspace*{-1em}
\subsection{Streaming data}
\label{sec:stream-data}



\vspace*{-1em}
\subsection{Reinforcement learning}
\label{sec:reinforce-learn}



\vspace*{-1em}
\subsection{Partial differential equation identification}
\label{sec:pde-identify}




\vspace*{-1em}
\subsection{Applications in data science related fields}
\label{sec:applications}

Symbiotically, data science theory informs new algorithms and methodologies, while the constraints of ``real-world'' applications  define new directions for fundamental exploration.
Within Georgia Tech the proximity to top researchers integrating inference and learning algorithms to solve real-world problems will  strengthen the impact of our foundational work, finding solutions that address practical barriers.



\vspace*{-1em}


\section{Institute Activities and Community Involvement}
The key goal of STRIDES institute is to not only establish a research program in the focus
areas at the foundations of data science described earlier, but also nurture and grow a
vibrant nationwide community around them, as well as undertake activities for community
benefit and training. Georgia Tech's excellent reputation and its established research and
educational programs in statistics, theoretical computer science (TCS), and mathematics,
along with our vast networks of existing collaborations, will help accomplish this goal.


%We describe details of our proposed activities below.
The following activities include workshops,
innovation labs,
short-term visitors,
graduate student and post-doc enabled joint research,
course development, and
a lecture series.



\vspace*{-1em}

\subsection{Innovation Labs/Thinktanks}
\label{sec:idea-labs}
We propose to organize innovation labs bringing together researchers from TCS, Math, and Stat, taking into account both applied and foundational perspectives, working together to translate applied questions into foundational questions across all areas.
The organizers will determine the themes of the innovation labs per the research programs proposed above, and will approach experienced senior researchers in the field with respect to mentoring and supervision.
We will utilize the {\em Idea Lab} framework that has been developed and experimented successfully in other related proposal-fertilization activities.
In particular, these innovation labs will take a form similar to study group sessions in industrial problem identification. These events will bring together senior and junior researchers, from different disciplines, to focus on specific data science related questions with the goal of turning the questions into new directions for novel theory and computations, which in turn will advance the foundations of data science.
The focus areas are based on themes in data science, and the outcomes of the proposed focus activities can then be translated into subsequent collaboration and co-supervision of  PhD and postdoc projects; they will also form the basis for future focused workshops, visitation of external experts, or research proposals.
%These events will be multi-day intensive workshops, with a lot of interaction that can include also external experts to help lead the problem identification groups.
Teams in different areas can work in parallel during the innovation labs. Some groups will likely include industry and government contacts.

\medskip

\noindent
{\bf Experience.}
Georgia Tech already has excellent examples of generating think-tank activities.
ARC (Algorithms \& Randomness Center) was founded to create interdisciplinary expertise tackling cross-disciplinary and real-world problems with experts in foundational needs.
GT-MAP (Georgia Tech Mathematics and Applications Portal) facilitates Mathematics as an effective research partner of the broader community at Georgia Tech, and provides a stable entity where researchers from the campus community can present their work and share ideas.
%TOOK THIS OUT AS IT WILL MAKE GT-MAP LOOK WEEK.
%GT-MAP recently began the process of supporting graduate students working on the GT-MAP related research, and anticipate to have post-doctoral fellows in the future.

\vspace*{-1em}

\subsection{Short-term visitation}
\label{sec:short-term}

During Phase I, we will approach potential researchers who are interested in spending short
periods of time at STRIDES to help enhance the proposed research.
Due to budget limitations, we may start with short-term visits, with each researcher staying
1-4 weeks.  We will proactively seek opportunities for co-hosting, tapping into other resources for joint sponsorship. We will also encourage and host sabbatical visitors to
fully or partially associate with STRIDES.


\vspace*{-1em}

\subsection{Lecture Series in Mathematical Foundations of Data Science}
\label{sec:lectures}

STRIDES will invite high-profile researchers each year to give a series of lectures related to the foundations of data science. Specific examples of distinguished researchers we plan to
invite include:

i) Prof. Roman Vershynin (University of Michigan) who has been working for many years on the development of non-asymptotic theory of random matrices and its applications to a number of problems in data science (compressed sensing, covariance estimation, statistics of networks, etc);

ii) Prof. Gabor Lugosi (Pompeu Fabra University, Barcelona, Spain) will be asked to lecture on Elements of Combinatorial Statistics.
%He is giving lectures on this during July 2017 in the Probability summer school in Saint-Flour, France.

iii) Prof. Boaz Klartag (Tel Aviv University, Israel), an expert in convex geometry and analysis, has agreed to give several lectures for a non-expert audience. Galyna Livshyts and Tetali plan to work with the local colleges and help host the lectures at Spelman college in Atlanta.%Also linking to relevant Colloquia in Math,CS, Engineering, ML etc

The institute will throw open these events to other universities/colleges in the area, e.g., Georgia State University, Emory University, Spelman College, as well as webcast the lectures
if speakers so permit to reach nationwide audience. Atlanta is home to the largest concentration of colleges and universities in the South, including historically black colleges and universities (HBCUs).

%Selection of the invited speakers will be charged by the Executive Committee that is described in the Collaboration and Evaluation plan.
%Section \ref{sec:leadership}.
%the Collaboration and Evaluation Plan.

%\vspace*{-1em}
%
%\subsection{Support through present infrastructure}
%\label{sec:infrastructure}
%
%The interdisciplinary research Institute for Data Engineering and Science (IDEaS) at Georgia Tech was conceived to provide a solid backbone integrating theoretical foundations of data science with more applied computational components.
%IDEaS has begun to seed efforts very well aligned with the TRIPODS program, such as STRIDES.
%Georgia Tech has a strong tradition of successful collaboration across the institute on the foundations side: the interdisciplinary research center and thinktank ARC is an affiliated center with IDEaS, so is the new Machine Learning Center (ML@GT).% (which was recently promoted to an Interdisciplinary Research Center by the EVPR.




\vspace*{-1em}

\section{Broader Impacts}
\label{sec:broaderimpact}

\vspace*{-1em}

\subsection{Enabling Interdisciplinary Collaboration}
One of the main objectives of the cross-disiplinary center will be fostering new collaborations across traditional disciplinary boundaries.  Many of the PIs and senior personnel have strong track records for promoting interdisciplinary research and understand the challenges.  We also have a lot of experience with data science and the effort required to ensure the theroetical questions pursued are of actual practical significance in the application domains.  Educational modules will be integrated into all of STRIDES's activities, such as prefacing all workshops with tutorials and taking extra efforts to enusre their appropriateness for newcomers to various fields.  We will provide means for interdisciplinary groups to have regular contact so that research forms a more solid interdisciplinary foundation.




\subsection{Communications}
\label{sec:communication}
An experienced Director of Communications will play a central role in
ensuring the success of STRIDES. During Phase I, STRIDES will be supported by
IDEaS communication director Jennifer Salazar. She has a record of success working with researchers on large interdisciplinary and multi-institution projects, possess rich experience leading research-related communication campaigns, digital media, and public relations, and is a talented writer. She will work closely with the Executive Leadership team to establish relationships, and to track achievements both internally and externally.
Her primary responsibilities will be to proactively remain apprised of important successes, develop a strategic communication plan and %communication
product calendar, and drive awareness of project progress both internally across the wider team, and externally to the research community, stakeholders, and general public.
%He or she will also collaborate with the leads of the project to establish and maintain effective outreach mechanisms.
We expect sustaining planned Phase II activity will require STRIDES hiring or at least
cost-sharing such a position.



\vspace*{-1em}

\subsection{Recruiting students and faculty to STRIDES programs}
\label{sec:recruit}

Georgia Tech's track record with recruiting students to their online and degree programs (e.g., the 25-year old, highly visible, ACO Ph.D. program) assures us that once advertised these courses and programs will be oversubscribed.
We will advertise on STRIDES web portal and other Georgia Tech websites as we develop new short courses and webinars.
We will  actively recruit students at large, including underrepresented minorities and women, and faculty from smaller colleges, to participate in our programs and/or apply for graduate fellowships and assistantships.
%We will send emails to recruit faculty from smaller colleges and community colleges.

\vspace*{-1em}
\subsection{Interaction with domain experts}

Due to strong presence of engineering programs and local data science industry, we have an efficient and effective interface to application domains.
Domain experts will be made aware of the developments and advances that theoretical foundations of data sciences have to offer.
The proposed institute will incorporate deep and frequent interactions between
theoreticians and domain experts.
%PT: suppressing the following comment, since what follows conveys the message.
%Algorithms developed in a vacuum for theoretical purposes only will typically fail to take into account the peculiarities and incompleteness properties of real data; the interactions enabled by STRIDES directly address this issue.
Success of theoretical foundations of data sciences will strongly depend on connections between statistical accuracy and quality-of-approximation as a tradeoff of various computational constraints that are imposed by
modern computing infrastructure; The fact that STRIDES will also be co-located with the high-performance computing center will enable such connections.


\vspace*{1em}
\section{Results from prior NSF support}% \label{sec:prior}
\vspace*{-0.5em}

The PIs are each supported by multiple NSF awards during the prior five years. Some reflect existing collaborative strengths among the PIs and senior personnel.
We summarize a few.
\vspace*{0.25em}

\noindent$\bullet$
{\bf Xiaoming Huo} is supported by DMS-1613152, and DMS-1106940, Achieving spatial adaptation via inconstant penalization: theory and computational strategies, Aug 2011--July 2014, \$140,000.

{\bf Intellectual Merit.}
The PI investigated how to achieve adaptive functional estimation when the underlying model has inhomogeneous roughness.
Publications that were partially enabled by this project include \cite{YangBook2014,Xu-13-pakdd,Kim-12-jns,Bastani-13-TASE,Wang-13-sp,Kim-13-aml,Kim-14-aor,Kim-14-ejs, Kim-14-TASE,Zhang-14,Wang-15-cusum,Huo-15-technometrics,Debraj-12-icdcs,lu-12-TAC,Huo-15-technometrics}.

{\bf Broader Impacts.} Huo disseminated the research through multiple external talks. He trained three female Ph.D. students (one graduated and is now an assistant professor).

\vspace*{0.25em}
\noindent
$\bullet$
{\bf Srinivas Aluru} has been supported by 11 NSF grants, including 6 from CCF. We report on IIS-1247716/1416259, BIGDATA: Genomes Galore - Core Techniques, Libraries, and Domain Specific Languages for High-Throughput DNA Sequencing, Jan 2013--Dec 2017, \$2,000,000.

{\bf Intellectual Merit:} This project is led by the PI in partnership with Stanford and Virginia Tech. The PI's group developed parallel algorithms for a variety of string and graph based index and data structures prevalent in bioinformatics. The work received multiple recognitions: best student paper (Supercomputing 2015); first selected paper by ACM SIGHPC under the scientific reproducibility initiative; and selection as a benchmark for Student Cluster Competition (Supercomputing 2016).

{\bf Broader Impacts:} Research results are disseminated as software libraries (github.com/ParBLiSS). The PI ran three international workshops to lead community efforts for high-throughput sequence analytics. He also assisted NSF in organizing a U.S.-Japan Big Data PI meeting.

\vspace*{0.25em}
\noindent
$\bullet$
{\bf Prasad Tetali} was supported by DMS-1407657 (July 2014--June 2018, \$288,000), CCF-1415496 and CCF-1415498 (Mar 2014--Feb 2017, \$600,000).

{\bf Intellectual Merit}.
The projects funded novel directions in optimal transport and discrete and continuous optimization, inspired by concrete real-world problems \cite{gozlan2015characterization, gozlan2014kantorovich}. The team modeled industrial challenges arising from current-day needs, and identified the precise algorithmic and optimization tools needed to solve the corresponding issues \cite{christensen2017approximation}.
%The team also developed convex optimization methods that scale with the size of the data and are robust to data uncertainty.

{\bf Broader Impacts}.
Tetali and team trained six Ph.D. students (one female) and two postdocs, and co-hosted an Industry Day for theoreticians. The team developed a pilot expert system {\em Ask Minmax}, to help non-experts diagnose and identify commonly encountered optimization problems.

\noindent
$\bullet$
\textbf{Jianfeng Lu} has been supported by several NSF projects in the
past. The most recent and relevant one is NSF DMS Grant 1454939,
``CAREER: Research and training in advanced computational methods for
quantum and statistical mechanics'' (09/01/2015--08/31/2020).

\smallskip
\textbf{Intellectual Merits:} The research goal of the
project is to innovate and analyze efficient algorithms based on
advanced computational mathematics for electronic structure theory and
computational statistical mechanics, which will greatly advance the
scope of ab initio simulations with applications in chemistry,
materials science, and related fields.  Under the support, Lu has been
highly productive and have already accomplished $83$ published or
accepted journal publications \cite{DelgadilloLuYang:16, LinLu:16,
  LiLu:16, LuYing:16f, LiLuSun:17, YuLuAbramsVandenEijnden:16,
  LuWirthYang:16, LuYing:16, LaiLu:16, LiLuYang:15, LuYing:15,
  LuZhou:16, CornelisYang:17, LiLinLu:18, LuYang:17,
  MendlLuLukkarinen:16, NiuLuoLuXiang:17, LiLuSun:17v, GaoLiuLu:17,
  LuThicke:17, GaoLiuLu:17w, WatsonLuWeinstein:17,
  LinLuVandenEijnden:18, LuYang:17c, XLiLu:17, LiLuYang:17, QLiLu:17,
  LuZhou:18, DaiLiLu:18, LuSteinerberger:17, YuCorsetti:18,
  LiLiuLu:17, LiLiuLuZhou:18, LuThicke:17c, GaucklerMarzuola:19,
  CaoLu:17, LaiLu:18, DuLiLuTian:18, HuangLuMing:18, LuZhou:18a,
  CaiLu:18surface, DelgadilloLuYang:18, ChenLuOrtner:18, LuYang:18,
  CaiLu:18qkmc, CaoLu:18, LuSteinerberger:18, BarthelLu:18, FangLu:18,
  YouLiLuGe:18, MartinssonLu:19, LuSachsSteinerberger:19+,
  WangLiLu:19, LuSoggeSteinerberger:19, ChenLiLu:19+,
  LuSteinerberger:19+, CaiLuYang:19+, LuSpiliopoulos:18, AnLuYing:19+,
  CaoLuLu:19FP, CaoLuLu:19Lindblad, ChenLiLuWright:19+, HuangLiuLu:19,
  KhooLuYing:19, LiLuWang:19, LiLu:19, LiuLuMargetisMarzuola:19,
  LuLuNolen:19, LuVandenEijnden:19, LuWang:19+, NishimuraDunsonLu:19+,
  ZhuQiuWang:19+, CaoLu:19+} and submitted manuscripts
\cite{CaiLuStubbs, KhooLuYing:uq, LiChengLu, LuOtto,
  LuWatsonWeinstein, ChenLiLuWright:b, LuSteinerberger, LiLuWang,
  ChenLiLuWright, SenSachsLuDunson, AgazziLu, LuLuNolen, ChenLiLu,
  LiLuMao, GaoLiuLuMarzuola, AnHead-GordonLinLu, LiLiLiuLiuLu,
  HolstHu, LiLuMattinglyWang, LuZhou, ThickeWatsonLu, GeLeeLi}.


\smallskip
\textbf{Broader Impacts:} During the  project, the PI has advised $10$ undergraduate students for
summer research projects ($4$ of them female) and is currently
advising $3$ graduate students working on this project. In addition,
the PI has advised a year-long research project with a team consists
of $5$ high school students ($2$ of them female) from North Carolina
School of Science and Mathematics.  Moreover, the PI
has co-organized and taught a MSRI-LBNL Summer School on Mathematical
Introduction to Electronic Structure Theory and taught in an IPAM
Summer School on Putting Back the Theory into the Density Functional
Theory. These summer schools attracted in total around $100$ graduate
students and early career researchers from US and abroad.

The PI has presented the results obtained during the project in many
colloquium talks, seminar talk and invited conference
presentations. The research results are further disseminated through
interactions and collaborations with computational chemists and
materials scientists. The PI has also organized many workshops and
conferences related with the project, including the KI-Net Conference
on Mathematical and Computational Methods in Quantum Chemistry, May
2016; SAMSI Workshop on Trends and Advances in Monte Carlo Sampling
Algorithm, December 2017; the 42nd SIAM Southeastern Section
Conference, March 2018; CSCAMM Workshop on Mathematical and Numerical
Aspects of Quantum Dynamics, June 2018. These conferences and
workshops bring together experts and young researchers across
disciplines.



%%%%%%%%%%%%%%%%%%%%%%

\clearpage


\begin{center}
{\bf\large II. Schedule}
\end{center}

(The next three parts should be no more than 10 pages.)

A tentative {\bf schedule} of scientific activities, with plans for Year 1 and a provisional schedule for Years 2 and 3.


\clearpage


\begin{center}
{\bf\large III. Plans for Human Resource Development}
\end{center}


{\bf Plans for human resource development}, including the selection and mentoring of a diverse cohort of students and postdoctoral participants, as appropriate, and the selection and involvement of researchers at all career levels.


\subsection{Graduate students and Postdocs}
\label{sec:gra-postdoc}

In Phase I, STRIDES has budgeted two graduate research assistantships and a partially sponsored postdoctoral fellowship.
We believe that collaboration can be more productive through cross-supervision; the funding will foster cross-collaborations that take a future-looking approach to recruitment and research training.

\medskip
\noindent
{\bf Cross-disciplinary graduate/postdoc co-supervision.}
The graduate research assistantship will support students to be co-advised across academic units on campus.
Supervisors must commit to be involved in co-supervision.
Students will work on trans-disciplinary projects.
The students learn a new field, bring that knowledge/connection back to home group, and bring new expertise to the other sponsoring group.
Fellows with advanced preparation and degrees will be admitted in a postdoc position.
The joint advising structure with be similar, with additional responsibilities.


Georgia Tech already has several interdisciplinary graduate programs.
%, such as the cross-campus master programs in statistics, quantitative and computational financed, etc, and cross-unit PhD programs such as Algorithms Randomness and Combinatorics, machine learning, and more.
A new Ph.D. program in Machine Learning is recently approved and will start running in 2018.
Another interdisciplinary Ph.D. program that has thrived for 25 years is the Algorithms, Combinatorics and Optimization (ACO) program, run jointly by the schools of mathematics, computer science and industrial and systems engineering, cutting across the Colleges of Sciences, Computing and Engineering.
STRIDES participants have ample experience with cross-disciplinary training.
They have a strong track record of mentoring and placement: e.g.,
Ruta Mehta and Jugal Garg (ARC postdocs, now faculty in CS and OR at UIUC, Illinois),
Phillipe Rigollet (Math postdoc, now faculty at MIT),
Will Perkins (NSF postdoc, faculty at Birmingham, U.K., moving to University of Waterloo, Canada),
Kevin Costello (NSF postdoc, faculty at UC-Riverside);
the first batch of NSF-funded IMPACT postdocs Maryam Yashtini and Christina Frederick soon to start faculty appointments at Georgetown and NJIT, respectively;
Stas Minsker (Math/Stats student, faculty at USC, LA),
Nayantara Bhatnagar, Adam Marcus, Amin Saberi and Emma Cohen (all ACO students, now faculty at Delaware, Princeton and Stanford, and a researcher at the Center for Communications Research in Princeton, respectively).



\subsection{Curriculum Development}
\label{sec:courses}

STRIDES will support the development of transdisciplinary courses for foundations of data
science. Justin Romberg and Mark Davenport will design a new data science related machine
learning course. Tetali and others will develop ``Mathematics of Data Science'' at the undergraduate level.
Planned topics include fundamentals from high-dimensional subspaces,
% (Law of Large Numbers, elements of gaussian random variables, concentration of measure for the sphere, random projections, and the Johnson-Lindenstrauss Lemma),
singular value decomposition,
%(Best k-rank approximation; methods for computing SVD, and applications)
Markov chains, and machine learning.
%(Basic examples; penalizing complexity; Perceptron Algorithm; Kernel Functions; VC-Dimension)
The institute will support activities whose benefits transcend Georgia Tech through
development of tutorials, on-line courses, course modules, lecture notes, tutorials, and
sharable slide sets.

This line of activities is aligned with many existing efforts on Georgia Tech campus.
We describe a select few.
Michael Lacey taught a special topics course on
``Mathematics of Compressive Sensing" during  Fall 2016, which had a large (50+) attendance.
%The topics covered the basics of linear compressive sensing, including concentration of measure for empirical processes, and their application to the analysis of random sensing matrices.
%The Null Space Property characterizes recovery of sparse vectors, and it's stability analogs.  Concentration of measure for empirical processes, and their application to the analysis of random sensing matrices.  VC Classes, and their role in the analysis of non-linear compressive sensing.  Elements of matricial compressive sensing.
Jeff Wu and Arkadi Nemirovski are currently co-teaching a course  on topics at the interface of statistics and optimization.
Arkadi Nemirovski, Vladimir Koltchinskii, and others are writing a book on optimization methods in statistics. Georgia Tech also has outstanding on-line degree programs hailed as
a national model.


\clearpage

\begin{center}
{\bf\large IV. Plans for Outreach and for Dissemination of Outcomes}
\end{center}

{\bf Plans for outreach and for dissemination of outcomes}.

\subsection{Workshops}
\label{sec:workshops}
STRIDES will hold about three workshops each year, along the lines of the three research
themes the institute focuses on. The workshops will be national gatherings of junior and senior researchers, early career researchers such as postdocs, and graduate and senior
undergraduate students, with additional international participants.
These will be week-long events which will include (besides research presentations)
tutorials, poster sessions, panels, working groups, open problem sessions, and
% suppressing -- speed-dating meetings ...
tea-time gatherings to encourage collaboration among researchers.
STRIDES will use advanced information technology to disseminate findings from these workshops online.

Our team has planned a number of initial workshops to hit the ground running as soon as the
institute is formed:

1. With the help of the Algorithms \& Randomness Center (ARC), we propose a workshop
on {\em Randomness in Data Science and Optimization}, and feature top experts in
TCS and optimization. This will be co-organized by
Singh, Vempala, Vigoda and Tetali.

2.  Koltchinskii, Nemirovski, and Romberg will co-organize a workshop addressing current challenges in optimization algorithms for modern datasets,  high-dimensional statistics and nonconvex inference problems, bringing together relevant statisticians and experts in continuous optimization and signal processing.

3. Huo and his colleagues (e.g., Le Song) will organize a workshop on decentralized and scalable statistical inference, as well as deep learning related methodologies.

\medskip

Topics for other workshops will be decided by the STRIDES team, taking community feedback and
evolving data science landscape into account for maximum impact. External researchers will be welcomed and involved in both organizing the scientific program of the workshops as well as
plan topic areas and activities. By leveraging support from synergistic institutions at Georgia Tech (IDEaS, ARC, South Hub, etc.), we will be able to organize or co-organize more workshops than the nine budgeted. We will also hold one or two workshops focused on
education in foundations of data science, vital to creating future workforce in this economically important area.

\noindent
{\bf Prior experience:}
The PIs have track record of hosting interdisciplinary workshops, some of which have launched new subfields in recent years. ARC hosted several workshops bringing experts in
randomized algorithms, MCMC and phase transitions, submodular optimization and network
science. Externally, Tetali co-organized a thematic workshop  {\em Graphical Models, Statistical Inference and Algorithms} (GRAMSIA) twice at the NSF-funded math institutes
IPAM (January 2012) and IMA (May 2015). Koltchinskii co-organized a workshop at the
NSF-funded institute SAMSI in 2014 on {\em Geometric Aspects of High-Dimensional Inference}.
Huo organized similar events at SAMSI, and is currently involved in the forthcoming
Joint Statistics Meeting, and a Banff workshop on data science related topics.
These workshops have been instrumental in bringing relevant experts  to come together and exchange novel ideas and breakthrough algorithms. The themes have evolved along with the frontier topics at the heart of the foundations of data science.

\clearpage

\baselineskip=1.2
\normalbaselineskip
\pagenumbering{arabic}

%\nocite{*}

\bibliographystyle{plain}
%\bibliographystyle{plainnat}
\bibliography{strides00,jl}



\end{document}

